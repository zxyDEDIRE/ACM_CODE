%!TEX program = xelatex
\documentclass[12pt, a4paper, oneside]{ctexart}
\usepackage[utf8]{inputenc}
\usepackage{ctex} %导入中文包
\usepackage{listings}
\usepackage{fontspec}
\usepackage{geometry} %设置页边距的包
\usepackage{listings}
\usepackage{xcolor} 

\usepackage{graphicx}
\usepackage{float}


\title{\fontsize{70}{30}\selectfont  ACM 算法模板} 
\author{Buns\_out} 
\date{\today} 

\geometry{left=2.5cm,right=2cm,top=2.54cm,bottom=2.54cm} %设置书籍的页边距
\definecolor{mygray}{rgb}{0.97,0.97,0.97}%定制颜色

\setsansfont{Fira Code}
\setmainfont{Fira Code}


%对于lstset排版
\lstset{
	tabsize=4,
	breaklines, 					% 自动将长的代码行换行排版
	backgroundcolor = \color{white},     			 % 背景色:淡黄
	numbers=left, 									% 行号在左侧显示
	numberstyle= \small, 								% 行号字体
	keywordstyle= \color{ red!70},						 % 关键字颜色
	commentstyle= \color{red!50!green!50!blue!50}, 		% 注释颜色
	rulesepcolor= \color{ red!20!green!20!blue!20} ,
	frame=single,                               % 设置代码框形式
	escapeinside=``,									 % 英文分号中可写入中文
	xleftmargin=2em,xrightmargin=1em, aboveskip=1em,
	framexleftmargin=2em
} 
% 题目编号变成英文
\renewcommand\thesection{\Alph{section}}

\usepackage{hyperref}
\hypersetup{
	hidelinks,
	colorlinks=true,
	allcolors=red!40!green!30!blue!100,
	pdfstartview=Fit,
	breaklinks=true
}


\begin{document} 



\centering


%\centering

\newpage
\lstset{language=C++}
% \qquad   

%############################################################################################################
\newpage 
\section{玉桑的环星球} 
\flushleft
出题人:爆哥 \par
考察点:数学推导,解方程 \par
\hspace*{\fill} \par
首先我们对于这个题,忽略掉第 $i$ 天走 $i^2$ 步这个条件,假定要是一天走一步的话,那么我们可以分析,对于不同的星球,每次回到了自己原来的出生点,那么必然是已经走了 $k*A_i\ (k\ge1)$ 步,那么对于 $n$ 个星球的话,他们要同时满足走 $X$ 步回到各自出生点,那么就是 $k_1*A_1=k_2*A_2...=k_n*A_n=X$ 那么就发现最小的 $X=LCM(A_1,A2,...,A_n)$ \par
\hspace*{\fill} \par
现在开始用 $lcm$ 代替 $LCM(A_1,A_2,...,A_n)$ \par 
\hspace*{\fill} \par
然后继续回到问题,那么我们可以得出结论:走的步数最后一定是这些数的最小公倍数的倍数。 \par
\hspace*{\fill} \par
那么我假设最少走了 $m$ 天之后 便可以满足他们各自重新回到了出生点,根据平方和公式得出 $m$ 天走了  $\frac{m*(m+1)*(2m+1)}{6}$ 步,那么根据前面推导,我们可以得出 $\frac{m*(m+1)*(2m+1)}{6}=k*lcm\ (k\geq1)$  \par
\hspace*{\fill} \par
移项之后 $m*(m+1)*(2m+1)=6*lcm*k\ (k\geq1)$ \par 
\hspace*{\fill} \par
然后这个时候我们可以得出 $m*(m+1)*(2m+1)$ 是 $6*lcm$ 的倍数这个结论 \par 
\hspace*{\fill} \par
然后此时你可以发现 $m\ m+1 \ 2m+1$  这三个数俩俩互质,这一点可以通过求辗转相减求最大公约数得出。 \par
\hspace*{\fill} \par
这个时候我们假设 $6*lcm$ 一共有 $q$ 个质数,第 $i$ 个质数有 $b_i$ 个 那么 $6*lcm=\prod_{i=1}^{q} p_i^{b_i}$ \par 
\hspace*{\fill} \par
那么通过上面得出的互质结论,每一个 $p_i^{b_i}$ 是 $m\ m+1 \ 2m+1$ 其中一个的因子 \par 
\hspace*{\fill} \par
然后题目给定 $lcm<=10^{12}$ 那么 $6*lcm<10^{13}$ 所以 $q$ 最大不超过 $11$  \par 
\hspace*{\fill} \par
那么我们可以暴力枚举每一个 $p_i^{b_i}$ 是这三个数中的哪一个因子 \par 
\hspace*{\fill} \par
假设暴力分配后的为 $a_1,a_2,a_3$ ,那么就有 $a_1*k_1=m,a_2*k_2=m+1,a_3*k_3=2m+1\ (k_1,k_2,k_3\geq1)$ \par 
\hspace*{\fill} \par
那么接下来只要解这三个方程即可  \par 
时间复杂度 $O(3^{11}log(\lambda))$   \par 
\begin{lstlisting}
#include<bits/stdc++.h>
using namespace std;
using ll=__int128;
ll Lcm(ll c,ll x){
    if(!c) return x;
    return c/__gcd(x,c)*x;
}
ll exgcd(ll a,ll b,ll &x,ll &y){
    if(!b){
        x=1;y=0;
        return a;
    }
    ll x1,y1,d;
    d=exgcd(b,a%b,x1,y1);
    x=y1;y=x1-a/b*y1;
    return d;
}
const int N=25;
ll pr[N];
int cnt[N];
int top;
ll ans;
ll res[3];
void dfs(int id){
    if(id==top){
        ll x,y;
        exgcd(res[0],res[1],x,y);
        ll k1=-x/res[1],k2=y/res[0];
        while(x+res[1]*k1>=0) k1--;
        while(y-res[0]*k2<=0) k2--;
        k1=min(k1,k2);
        ll c=res[0]*(-x)+y*res[1];
        ll a=2ll*res[0]*res[1],b=res[2];
        ll x1,y1;
        ll d=exgcd(b,a,x1,y1);
        if(c%d) return;
        x1*=c/d;y1*=c/d;
        ll b1=b/d,a1=a/d;
        ll k3=-x1/a1;
        ll k4=(y1-k1)/b1;
        while(y1-k4*b1>k1) k4++;
        while(x1+k3*a1<=0) k3++;
        k1=y1-max(k4,k3)*b1;
        ans=min(ans,-(x+res[1]*k1)*res[0]);
        return;
    }
    for(int i=0;i<3;i++)
    {
        res[i]=res[i]*pr[id];
        dfs(id+1);
        res[i]=res[i]/pr[id];
    }
}

long long x[N];
void solve(){
    int n;
    cin>>n;
    ll lc=0;
    for(int i=1;i<=n;i++){
        cin>>x[i];
        lc=Lcm(lc,x[i]);
    }
   
    top=0;
    lc*=6;
    ans=lc;
    for(int i=2;1ll*i*i<=lc;i++){
        if(lc%i==0){
            pr[top]=1;
            cnt[top]=0;
            while(lc%i==0){
                lc/=i;cnt[top]++;
                pr[top]*=i;
            }
            top++;
        }
    }
    if(lc>1){
        pr[top]=lc;
        cnt[top]=1;
        top++;
    }
    res[0]=res[1]=res[2]=1;
    dfs(0);
    cout<<(long long)ans<<'\n';
}


int main(){
    ios::sync_with_stdio(false);cin.tie(0);cout.tie(0);
    int t=1;
    while(t--){
        solve();
    }
    return 0;
}
\end{lstlisting}





\newpage 
\section{干饭豪的不定方程} 
出题人:爆哥  \par
考察点:暴力枚举 \par 
\hspace*{\fill} \par
对于 $n<=10^9,a,t,b,k\geq2$ 那么我们可以发现单暴力枚举 $x^y$ 这种形式不会有很多,所以我们可以先预处理出来这些情况方案之后,在对于每个询问,枚举 $b^k$ 来看之前多少种 $a^t=n-b^k$ 情况 \par 
\begin{lstlisting}
/*
干饭豪的a^k+b^t=n
数据未更新
*/
/*
    有多少个四元组(a,t,b,k)满足下面式子
    a^t+b^k=n
    且a,b,t,k>=2
    n<=1e9
*/
#include<bits/stdc++.h>
using namespace std;

void solve(){
    int n;
    cin>>n;
    long long ans=0;
    map<long long,int>mp;
    mp.clear();

    for(int a=2;;a++){
        long long c=1ll*a*a;
        if(c>=n) break;
        for(int t=2;c<n;t++,c=c*a){
            mp[c]++;
        }
    }
    for(int a=2;;a++){
        long long c=1ll*a*a;
        if(c>=n) break;
        for(int t=2;c<n;t++,c=c*a){
            if(mp.count(n-c)){
                ans+=mp[n-c];
            }
        }
    }
    cout<<ans<<'\n';
}
int main(){
    ios::sync_with_stdio(false);cin.tie(0);cout.tie(0);
    int t=1;
    cin>>t;
    while(t--){
        solve();
    }
    return 0;
}
/*
*/
\end{lstlisting}


\newpage 
\section{狗头豪的数列求和} 
出题人:爆哥 \par 
考察点:数学推导,矩阵快速幂 \par 
本题需要求 $1^k+2^k+...+n^k\ (n\leq10^{18})$ \par 
\hspace*{\fill} \par
发现 $n$ 可能会很大不能正常的通过遍历来求和,那么我们来思考每一位 \par 
\hspace*{\fill} \par
定义 $sum(x,k)$ 为前 $x$ 项 $k$ 次方的和 \par 
\hspace*{\fill} \par
首先可以得出 $sum(x,k)=sum(x-1,k)+x^k \ (x\geq2)$ \par 
\hspace*{\fill} \par
发现 $sum(x,k)$ 是用一个递推关系的,那么思考 $(x-1)^k$ 和 $x^k$ 会有什么关系  \par 
\hspace*{\fill} \par
可以通过高中学过的二项式展开得到 $(x-1)^k=\sum_{i=0}^kC_k^ix^i(-1)^{k-i}$ \par 
\hspace*{\fill} \par
得出 $x^k=(x-1)^k-\sum_{i=0}^{k-1}x^i(-1)^{k-i}$ \par 
\hspace*{\fill} \par
那么我们发现这显然是一个线性递推关系的 \par 
\hspace*{\fill} \par
$k<=10$ 那么我们可以维护一个矩阵的信息 分别处理 $1,x^1,x^2,...,x^k,sum(x,k)$ 的递推联系 \par 
\hspace*{\fill} \par
便可以通过矩阵快速幂来解决这道题 \par 
\hspace*{\fill} \par
时间复杂度 $O(k^3log(n))$ \par 
\hspace*{\fill} \par
当然,如果你知道拉格朗日插值法,那么这个题就变的很裸,不过要注意这个题模数不一定会有逆元,就看你通过什么方式来处理了。 \par
\begin{lstlisting}

#include<bits/stdc++.h>
using namespace std;
const int N=15;
using ll=long long;
int k,mod;
long long n;
struct mat{
	int m;
	ll mp[N][N];
	mat(){
		memset(mp,0,sizeof(mp));
	}
	void re(){
		for(int i=0;i<m;i++){
			mp[i][i]=1;
		}
	}
	mat operator *(const mat &w)const{
		mat res;
		res.m=m;
		for(int k=0;k<m;k++){
			for(int i=0;i<m;i++){
				for(int j=0;j<m;j++){
					res.mp[i][j]+=mp[i][k]*w.mp[k][j]%mod;
					res.mp[i][j]%=mod;
				}
			}
		}
		return res;
	}
};

mat ksm(mat a,long long b){
	mat res;
	res.m=a.m;
	res.re();
	while(b){
		if(b&1) res=res*a;
		a=a*a;
		b>>=1;
	}
	return res;
}

ll c[N][N];

void solve(){
	cin>>n>>k>>mod;
	int m=k+2;
	for(int i=0;i<=m;i++){
		for(int j=0;j<=i;j++){
			if(i==j||j==0){
				c[i][j]=1;
			}
			else{
				c[i][j]=c[i-1][j-1]+c[i-1][j];
				if(c[i][j]>=mod) c[i][j]-=mod;
			}
		}
	}
	mat a;
	a.m=m;
	for(int j=m-1;j>=1;j--){
		a.mp[j][j]=1;
		int k1=k-j+1;
		for(int i=j+1,k2=k1-1,f=1;i<m;i++,k2--,f=mod-f){
			for(int p=i;p<m;p++){
				a.mp[p][j]+=f*c[k1][k2]%mod*a.mp[p][i];
				a.mp[p][j]%=mod;
			}
		}
	}
	a.mp[0][0]=1;
	for(int i=1;i<m;i++){
		a.mp[i][0]=a.mp[i][1];
	}
	mat res;
	res.m=m;
	for(int j=0;j<m;j++){
		res.mp[0][j]=1;
	}
	res=res*ksm(a,n-1);
	cout<<res.mp[0][0]<<'\n';
}

int main(){
	ios::sync_with_stdio(false);cin.tie(0);cout.tie(0);
	int t=1;
	while(t--){
		solve();
	}
	return 0;
}

\end{lstlisting}




\newpage 
\section{草莓豪的斐波那契} 
出题人:爆哥 \par 
考察点:思维 \par 
\hspace*{\fill} \par
可以通过观察斐波那契序列值是指数形势增长,所以直接模拟即可 \par 
\begin{lstlisting}
#include<bits/stdc++.h>
using namespace std;
void solve(){
	long long n;
	cin>>n;
	long long now=0;
	long long a=1,b=1;
	int t=0;
	while(now<n){
		now+=b;
		t++;
		a=a+b;
		swap(a,b);
	}
	cout<<t<<'\n';
}
int main(){
	ios::sync_with_stdio(false);cin.tie(0);cout.tie(0);
	int t=1;
	while(t--){
		solve();
	}
	return 0;
}
\end{lstlisting}


\newpage 
\section{\href{https://ac.nowcoder.com/acm/contest/72386/H}{小刻觉得应该来个平衡树}} 
\href{https://ac.nowcoder.com/acm/contest/72386/H}{小刻觉得应该来个平衡树}
\hspace*{\fill} \par
**宣传一下我的模板库:** 

[Release ACM模板 · zxyDEDIRE/ACM-Code-Library (github.com)](https://github.com/zxyDEDIRE/ACM-Code-Library/releases/tag/QAQ)

[zxyDEDIRE/ACM-Code-Library at QAQ (github.com)](https://github.com/zxyDEDIRE/ACM-Code-Library/tree/QAQ)

有LaTeX源码也有PDF版本
\begin{lstlisting}
\end{lstlisting}



\end{document}

